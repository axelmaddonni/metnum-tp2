\subsection{Apéndice A}


\vskip 5pt
\noindent\textbf{Enunciado}
\vskip 5pt

El objetivo del trabajo es experimentar en el contexto planteado utilizando el algoritmo PageRank con las variantes propuestas. A su vez, se busca
comparar los resultados obtenidos cualitativa y cuantitativamente con los algoritmos tradicionales utilizados en cada uno de los contextos planteados. 
Los m\'etodos a implementar (como m\'inimo) en ambos contexto planteados por el trabajo son los siguientes:

\begin{enumerate}
\item \emph{B\'usqueda de p\'aginas web:} PageRank e \textsc{In-deg}, \'este \'ultimo consiste en definir el ranking de las p\'aginas utilizando 
solamente la cantidad de ejes entrantes a cada una de ellas, orden\'andolos en forma decreciente.
\item \emph{Rankings en competencias deportivas:} GeM y al menos un m\'etodo est\'andar propuesto por el grupo (ordenar por victorias/derrotas,
puntaje por ganado/empatado/perdido, etc.) en funci\'on del deporte(s) considerado(s).
\end{enumerate}

El contexto considerado en 1., en la b\'usqueda de p\'aginas web, representa un desaf\'io no s\'olo desde el modelado, si no tambi\'en desde el punto 
de vista computacional considerando la dimensi\'on de la informaci\'on y los datos a procesar. Luego, dentro de nuestras posibilidades, consideramos
un entorno que simule el contexto real de aplicaci\'on donde se abordan  instancias de gran escala (es decir, $n$, el n\'umero total de p\'aginas, es 
grande). Para el desarrollo de PageRank, se pide entonces considerar el trabajo de Bryan y Leise \cite{Bryan2006} donde se explica la intuci\'on y algunos 
detalles t\'ecnicos respecto a PageRank. Adem\'as, en Kamvar et al. \cite{Kamvar2003} se propone una mejora del mismo. Si bien esta mejora queda fuera de 
los alcances del trabajo, en la Secci\'on 1 se presenta una buena formulaci\'on del algoritmo. En base a su definici\'on, $P_2$ no es una matriz esparsa. 
Sin embargo, en Kamvar et al. \cite[Algoritmo 1]{Kamvar2003} se propone una forma alternativa para computar $x^{(k+1)} = P_2 x^{(k)}$. Este resultado debe 
ser utilizado para mejorar el almacenamiento de los datos.

En la pr\'actica, el grafo que representa la red de p\'aginas suele ser esparso, es decir, una p\'agina posee relativamente pocos links de salida comparada 
con el n\'umero total de p\'aginas. A su vez, dado que $n$ tiende a ser un n\'umero muy grande, es importante tener en cuenta este hecho a la hora de definir 
las estructuras de datos a utilizar. Luego, desde el punto de vista de implementaci\'on se pide utilizar alguna de las siguientes estructuras de datos para 
la representaci\'on de las matrices esparsas: \emph{Dictionary of Keys} (dok), \emph{Compressed Sparse Row} (CSR) o \emph{Compressed Sparse Column} (CSC). 
Se deber\'a incluir una justificaci\'on respecto a la elecci\'on que consdiere el contexto de aplicaci\'on. Adem\'as, para PageRank se debe implementar el 
m\'etodo de la potencia para calcular el autovector principal. Esta implementaci\'on debe ser realizada \'integramente en \textsc{C++}.

En funci\'on de la experimentaci\'on, se deber\'a realizar un estudio particular para cada algoritmo (tanto en t\'erminos de comportamiento
del mismo, como una evaluaci\'on de los resultados obtenidos) y luego se proceder\'a a comparar cualitativamente los rankings generados.
La experimentaci\'on deber\'a incluir como m\'inimo los siguientes experimentos:
\begin{enumerate}
\item Estudiar la convergencia de PageRank, analizando la evoluci\'on de la norma Manhattan (norma $L_1$) entre dos iteraciones sucesivas. Comparar los 
resultados obtenidos para al menos dos instancias de tama\~no mediano-grande, variando el valor de $c$. 
\item Estudiar el tiempo de c\'omputo requerido por PageRank. 
\item Para cada algoritmo, proponer ejemplos de tama\~no peque\~no que ilustren el comportamiento esperado (puede ser utilizando las herramientas provistas
por la c\'atedra o bien generadas por el grupo).
\end{enumerate}

Puntos opcionales:
\begin{enumerate}
\item Demostrar que los pasos del Algoritmo 1 propuesto en Kamvar et al. \cite{Kamvar2003} son correctos y computan $P_2 x$.
\item Establecer una relaci\'on con la proporci\'on entre $\lambda_1 = 1$ y $|\lambda_2|$ para la convergencia de PageRank.
\end{enumerate}

El segundo contexto de aplicaci\'on no presenta mayores desaf\'ios desde la perspectiva computacional, ya que en el peor de los casos una liga no suele tener
mas que unas pocas decenas de equipos. M\'as a\'un, es de esperar que en general la matriz que se obtiene no sea esparsa, ya que probablemente un equipo juegue
contra un n\'umero significativo de contrincantes. Sin embargo, la popularidad y sensibilidad del problema planteado requieren de un estudio detallado y 
pormenorizado de la calidad de los resultados obtenidos. El objetivo en este segundo caso de estudio es puramente experimental. 

En funci\'on de la implementaci\'on, a\'un cuando no represente la mejor opci\'on, es posible reutilizar y adaptar el desarrollo realizado para p\'aginas web. 
Tambi\'en es posible realizar una nueva implementaci\'on desde cero, simplificando la operatoria y las estructuras, en \textsc{C++}, \textsc{Matlab} o 
\textsc{Python}.

La experimentaci\'on debe ser realizada con cuidado, analizando (y, eventualmente, modificando) el modelo de GeM:
\begin{enumerate}
\item Considerar al menos un conjunto de datos reales, con los resultados de cada fecha para alguna liga de algu\'un deporte.
\item Notar que el m\'etodo GeM asume que no se producen empates entre los equipos (o que si se producen, son poco frecuentes). En caso de considerar un 
deporte donde el empate se da con cierta frecuencia no despreciable (por ejemplo, f\'utbol), es fundamental aclarar como se refleja esto en el modelo y 
analizar su eventual impacto.
\item Realizar experimentos variando el par\'ametro $c$, indicando como impacta en los resultados. Analizar la evoluci\'on del ranking de los equipos a 
trav\'es del tiempo, evaluando tambi\'en la evoluci\'on de los rankings e identificar caracter\'isticas/hechos particulares que puedan ser determinantes 
para el modelo, si es que existe alguno.
\item Comparar los resultados obtenidos con los reales de la liga utilizando el sistema est\'andar para la misma.
\end{enumerate}

Puntos opcionales:
\begin{enumerate}
\item Proponer (al menos) dos formas alternativas de modelar el empate entre equipos en GeM.
\end{enumerate}


\vskip 5pt
\noindent\textbf{Par\'ametros y formato de archivos}
\vskip 5pt

El programa deber\'a tomar por l\'inea de comandos dos par\'ametros. El primero de ellos contendr\'a la informaci\'on del experimento, incluyendo
el m\'etodo a ejecutar (\verb+alg+, 0 para PageRank, 1 para el m\'etodo alternativo), la probabilidad de teletransportaci\'on $c$, el tipo de instancia
(0 p\'aginas web, 1 deportes), el \emph{path} al archivo/directorio conteniendo la definici\'on de la red (que debe ser relativa al ejecutable, o el path 
absoluto al archivo) y el valor de tolerancia utilizado en el criterio de parada del m\'etodo de la potencia. 

El siguiente ejemplo muestra un caso donde se pide ejecutar PageRank, con una probabilidad de teletransportaci\'on de 0.85, sobre la red descripta en 
\verb+test1.txt+ (que se encuentra en el directorio \verb+tests/+), correspondiente a una instancia de ranking aplicado a deportes y con una tolerancia 
de corte de $0.0001$.
\begin{verbatim}
0 0.85 1 tests/red-1.txt 0.0001
\end{verbatim}

Para la definici\'on del grafo que representa la red, se consideran dos bases de datos de instancias con sus correspondientes formatos. La primera
de ellas es el conjunto provisto en SNAP \cite{SNAP} (el tipo de instancia es 0), con redes de tama\~no grande obtenidos a partir de datos reales. Adem\'as, 
se consideran las instancias que se forman a partir de resultados de partidos entre equipos, para alg\'un deporte elegido por el grupo. 

En el caso de la base de SNAP, los archivos contiene primero cuatro l\'ineas con informaci\'on sobre la instancia (entre ellas, $n$ y la cantidad
total de links, $m$) y luego $m$ l\'ineas con los pares $i$, $j$ indicando que $i$ apunta a $j$. A modo de ejemplo, a continuaci\'on se muestra el 
archivo de entrada correspondiente a la red propuesta en Bryan y Leise \cite[Figura 1]{Bryan2006}: 

\begin{verbatim}
# Directed graph (each unordered pair of nodes is saved once): 
# Example shown in Bryan and Leise.
# Nodes: 4 Edges: 8 
# FromNodeId    ToNodeId
1   2
1   3
1   4
2   3
2   4
3   1
4   1
4   3
\end{verbatim}

Para el caso de rankings en ligas deportivas, el archivo contiene primero una l\'inea con informaci\'on sobre la cantidad de equipos ($n$), y la cantidad
de partidos totales a considerar ($k$). Luego, siguen $k$ l\'neas donde cada una de ellas representa un partido y contiene la siguiente informaci\'on: 
n\'umero de fecha (es un dato opcional al problema, pero que puede ayudar a la hora de experimentar), equipo $i$, goles equipo $i$, equipo $j$, goles equipo $j$.
A continuaci\'on se muestra el archivo de entrada con la informaci\'on del ejemplo utilizado en Govan et al. \cite{Govan2008}:

\begin{verbatim}
6 10
1 1 16 4 13
1 2 38 5 17
1 2 28 6 23
1 3 34 1 21
1 3 23 4 10
1 4 31 1 6
1 5 33 6 25
1 5 38 4 23
1 6 27 2 6
1 6 20 5 12
\end{verbatim}

Es importante destacar que, en este \'ultimo caso, los equipos son identificados mediante un n\'umero. Opcionalmente podr\'a considerarse un archivo que contenga, 
para cada equipo, cu\'al es el c\'odigo con el que se lo identifica.

Una vez ejecutado el algoritmo, el programa deber\'a generar un archivo de salida que contenga una l\'inea por cada
p\'agina ($n$ l\'ineas en total), acompa\~nada del puntaje obtenido por el algoritmo PageRank/\textsc{In-deg}/m\'etodo alternativo. 

Para generar instancias de p\'aginas web, es posible utilizar el c\'odigo Python provisto por la c\'atedra. La utilizaci\'on del mismo se
encuentra descripta en el archivo README. Es importante mencionar que, para que el mismo funcione, es
necesario tener acceso a Internet. En caso de encontrar un bug en el mismo, por favor contactar a los docentes de la
materia a trav\'es de la lista. Desde ya, el c\'odigo puede ser modificado por los respectivos grupos agregando todas
aquellas funcionalidades que consideren necesarias.

Para instancias correspondientes a resultados entre equipos, la c\'atedra provee un conjunto de archivos con los resultados del Torneo de Primera Divisi\'on 
del F\'utbol Argentino hasta la Fecha 23. Es importante aclarar que los dos partidos suspendidos, River - Defensa y Justicia y Racing - Godoy Cruz han sido 
arbitrariamente completados con un resultado inventado, para simplificar la instancia. En funci\'on de datos reales, una alternativa es considerar el 
repositorio DataHub \cite{datahub}, que contiene informaci\'on estad\'istica y resultados para distintas ligas y deportes de todo el mundo.

\vskip 5pt

%\subsection{Apéndice B}
%
%\textbf{Código fuente para ...}
%\textbf{}

%\lstset{language=C++, breaklines=true, basicstyle=\footnotesize}
%\begin{lstlisting}[frame=single]
%
%\end{lstlisting}
