\par Concluimos que el algoritmo PageRank es superior al In-Deg para establecer un ranking de p\'aginas web.
Tambi\'en concluimos que el tiempo de convergencia depende no s\'olo de la cantidad de elementos no-nulos de la matriz de transici\'on (es decir, la cantidad de v\'ertices en el grafo que representa a la red), sino de la cantidad de p\'aginas en la web representada (la cantidad de nodos en el grafo).
Adicionalmente, concluimos que el tiempo de convergencia aumenta al incrementar el par\'ametro \textit{''c''}.
\par Tambi\'en concluimos que nuestra variante del algoritmo PageRank de personalizaci\'on es efectiva, cuando se emplea un par\'ametro \textit{``p''} apropiado (alrededor del rango [0.75, 0.85].

\bigskip

\par Por parte de la aplicaci\'on del m\'etodo GeM en ligas deportivas podemos concluir que es un m\'etodo interesante para realizar comparaciones con los rankings est\'andar. Nos deja con ganas de experimentar sobre otras ligas o deportes y ver que resultados obtenemos. Por cuestiones de tiempo no pudimos llevar a cabo estos experimentos, pero sabemos que podemos ir ajustando los par\'ametros y variando algunos pasos del m\'etodo para lograr resultados diversos y enfocar el ranking en los datos que creamos importantes.
\par Pudimos notar que los tiempos de ejecuci\'on no fueron relevantes y no tuvimos necesidad de optimizar el proceso ni las estructuras de memoria, dado que los datos eran muy acotados.
\par 

\bigskip

\par En terminos generales pudimos tomar m\'etodos num\'ericos y aplicarlos sobre casos reales. Esto implic\'o tener cuidado con los costos (tanto de tiempo como de memoria) y verificar los resultados con respecto a los datos reales. No nos llevamos grandes sorpresas pero nos pareci\'o interesante y entretenido poder aplicar el TP a casos reales y cercanos a la vida diaria y ver como se comportaban.