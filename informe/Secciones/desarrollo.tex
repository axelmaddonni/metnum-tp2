\subsection{Page Rank para Páginas Web}

\subsubsection{Algoritmo PageRank vs IN-DEG}

\par El algoritmo de PageRank presentado cumple con lo especificado en la introducción teórica. Utilizando el método de la potencia explicado anteriormente, dada la matriz P'' calculada a partir del input, y un parámetro de teleportación $c$ pasado por parámetro, computa el primer autovector principal de probabilidad que muestra la "proporción" en el tiempo que un navegante aleatorio pasaría en cada página. A partir de ese vector, confeccionamos el Ranking que es devuelto en un archivo de salida.

\par Usando el Algoritmo propuesto por Kamvar, demostrado en la sección anterior, aprovechamos la esparsidad de la matriz P. Para ello utilizamos una estructura llamada \textit{Compressed Column Storage} para optimizar el tiempo de ejecución de nuestro algoritmo. La estructura que aprovecha esta característica se encuentra explicada en la siguiente sección.

\par Por otra parte, el algoritmo IN-DEG que también utilizaremos en nuestros experimentos para comparar computa un Ranking solamente tomando en cuenta los enlaces entrantes a una página, ordenándolas por importancia de manera decreciente.

	
\subsubsection{Representación de Matrices Esparsas}

\par En el enunciado de este trabajo se mencionaron tres estructuras distintas para aprovechar la esparsidad de la matriz $P$ (detalla por Kamvar et al\cite{Kamvar2003}).
\'Estas son: Dictionary of Keys (\textit{DoK}), Compressed Column Storage (\textit{CCS}) y Compressed Row Storage (\textit{CRS}).
\par \textit{DoK} consiste en guardar triplas de $<fila, columna, valor>$ para cada elemento no-nulo de la matriz. 
Permite construir la representaci\'on de la matriz incrementalmente y el acceso aleatorio r\'apido a los elementos de \'esta
(ya sea en tiempo $O(1)$ o $O(log(n))$ - con $n$ igual a la cantidad de elementos no nulos-, dependiendo de la implementaci\'on elegida para el diccionario).
Sin embargo, no es particularmente buena para realizar operaciones de matrices, y el overhead incurrido por esta estructura es mayor que el de las otras.
\par \textit{CCS} y \textit{CRS} mantienen tres arreglos: uno para los valores no-nulos (ordenados por columna y luego por fila en el caso de \textit{CCS} y al rev\'es en el caso de \textit{CRS}),
uno que indica a qu\'e columna (en el caso de \textit{CCS}, o fila para \textit{CRS}) corresponden esos valores y uno que determina c\'uantos elementos de cada fila 
(para \textit{CCS}, o columna para \textit{CRS}) son no-nulos.
\textit{CCS} es eficiente (itera s\'olo sobre los elementos no-nulos, y el overhead es $O(1)$) para multiplicar por derecha (es decir, con la matriz \textit{CCS} siendo la derecha), 
y \textit{CRS} para multiplicar por izquierda.
\par La matriz $P$ se puede construir a partir de la entrada en una \textit{CRS}, o se puede construir $P^T$ como \textit{CCS} (la representaci\'on de una matriz $A$ cualquiera en \textit{CRS}
es id\'entica a la representaci\'on \textit{CCS} de $A^T$). 
Nosotros necesitamos $P^T$ como \textit{CRS}, por lo que deberemos pasar la representaci\'on \textit{CCS} de $P^T$ a \textit{CRS}.
Afortunadamente, esto se puede realizar en tiempo lineal en la cantidad de elementos no-nulos (equivalente al tiempo necesario para transponer la matriz, por lo que no se incurre en overhead, desde
un punto de vista asint\'otico).
	
\subsection{Problema del Page Rank para Competencias Deportivas}

\subsubsection{Algoritmo GeM}

\par Para implementar el algoritmo GeM no tuvimos grandes problemas. Decidimos hacer una implementaci\'on paralela al Page Rank para evitar complicaciones, ya que sabemos que contamos con las siguientes caracter\'isticas:
\begin{itemize}
	\item Una cantidad de datos acotados: la matriz est\'a definida por la cantidad de equipos, pero sabemos que es acotada y que no nos va a generar grandes problemas.
	\item En la liga que vamos a analizar los equipos se enfrentar todos contra todos, por lo que no vamos a toparnos con una matriz esparsa. Por eso nos vamos a ahorrar las optimizaciones de estructuras (mas a\'un contando con el item anterior).
	\item Los resultados son n\'umeros enteros y de rango acotado (acotados por el juego en s\'i, ya que ser\'ia un caso borde que un equipo convierta mas de 6 goles en un partido).
\end{itemize}
\par Analizando los puntos nombrados decidimos tomar un vector de vectores de double (matriz de double ``M''). Vamos recorriendo el archivo de entrada donde se encuentran los resultados de los distintos partidos y aplicamos la siguiente l\'ogica:

\begin{algorithmic}[1]
\If{goles(j) == goles(i)}
	\State //salteamos el partido
\ElsIf{goles(j) \textgreater goles(i)}
	\State $M_{ij} = M_{ij} + ( goles(j) - goles(i) )$
\Else
	\State $M_{ji} = M_{ji} + ( goles(i) - goles(j) )$
\EndIf
\end{algorithmic}

Con este paso obtenemos la ``matriz $A^t$''. Luego la recorremos y dividimos a cada posici\'on $ij$ por la suma de todas las posiciones de su misma fila $i$. as\'i obtenemos la ``matriz $H^t$. Luego tomamos a $P$=$H^t$ y realizamos los pasos del m\'etodo Page Rank. Al finalizar imprimimos los resultados en el archivo de salida.

\subsubsection{M\'etodo Est\'andar}

	\par El m\'etodo est\'andar para la obtenci\'on del ranking de una liga de f\'utbol consta sumar los puntajes de los equipos en cada partido (3pts por ganar y 1pt por empatar). Para llevar esto a cabo tomamos un vector de enteros y recorremos el archivo de entrada aplicando el siguiente algoritmo para cada partido:

\begin{algorithmic}[1]
\If{goles(j) == goles(i)}
	\State $M_{ij} = M_{ij} + 1$
	\State $M_{ij} = M_{ji} + 1$
\ElsIf{goles(j) \textgreater goles(i)}
	\State $M_{ij} = M_{ij} + 3$
\Else
	\State $M_{ji} = M_{ji} + 3$
\EndIf
\end{algorithmic}

\par Al finalizar el proceso obtenemos el vector con los puntajes de cada equipo. Lo siguiente que hacemos es ordenarlo e imprimir los resultados en el archivo de salida.
